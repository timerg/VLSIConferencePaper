% Template for VLSI/CAD-2013 paper; to be used with:
%         vlsiconf.sty  - ICASSP/ICIP LaTeX style file, and
%          IEEEbib.bst - IEEE bibliography style file.
% --------------------------------------------------------------------------
\documentclass{article}
\usepackage{vlsiconf,amsmath,epsfig,anyfontsize}
\usepackage{graphicx}
\usepackage{transparent}
\usepackage{color}

% \pagecolor{blue}
\graphicspath{{Graphs/}}
% \setCJKmainfont[Mapping=tex-text]{STSongti-TC-Light}
% \setCJKsansfont[Mapping=tex-text]{STSongti-TC-Light}
% \setCJKmonofont[Mapping=tex-text]{STSongti-TC-Light}

%
% Example definitions.
% --------------------
\def\x{{\mathbf x}}
\def\L{{\cal L}}

% Title.
% ------
\title{An Integrated Circuit Design for Silicon-Nanowire Read Out Circuit}
%
% Single address.
% ---------------
\name{Author(s) Name(s)}
\address{Department of Electrical Engineering, National Tsing Hua University}
%

%
\begin{document}
%\ninept
%
\maketitle
%
\begin{abstract}
A measurement method for poly-silicon nanowire field-effect transistor (SiNW FET) is proposed and the read-out circuit for realizing the method is designed.
The method helps to mitigate the device mismatch problem.
The problem is common and mostly caused by the fabrication variance.
The method utilizes the linearity relation between device current ($I_{ds}$) and transconductance ($gm$).
During the measurement, the current of nanowire is kept in a fixed value, which means $gm$ of device is constant as well.
By controlling the $gm$ of mismatched devices to be same, the devices produce same output current when the testing bio-molecule concentration changes.
The current variance is then transformed into voltage and that voltage is amplified in the end.

This article shows the design concept, circuit schematic, table of specification and chip measurement results.
\end{abstract}
%
\section{Introduction}
\label{sec:intro}

Poly-silicon nanowire(SiNW) is an interesting one-dimensional nano-structures.
Many research of fabrication and electrical properties have been conducted \cite{C25th}.
Since it was first introduced to the biosensor field in 2001 \cite{C2001}, it has become a promising candidate for ultra-sensitive, real-time and label-free sensor device.

One of the challenges of nanowire is the device variability problem caused by fabrication variation.
The problem results in the device mismatch for devices fabricated in the same process.
For our nanowire (Fig.\ref{fig:draw}), the problem exists and effect the measurement severely.

In this work, a measurement method aimed to solve the problem is proposed, and the read-out circuit for realizing the method is designed.
The results presented in the end of this work show the method effectively mitigate the problem.


\begin{figure}[hbt]
    \centering
    {\fontfamily{pag}\selectfont\textbf{
        \def\svgwidth{5.0cm}
        \fontsize{6}{7}\selectfont
        \input {Graphs/drawing.pdf_tex}
    }}
    \caption{Nanowire device Structure}
    \label{fig:draw}
\end{figure}

\subsection{Constant-Current Constant-Transconductance}
For a simple MOSFET, its transconductance($gm$) is related to its drain-to-srouce current ($I_{ds}$).
\begin{align}
    \text{Strong inversion region: } & \sqrt{2I_{ds} (\kappa \mu C_{ox} \frac{W}{L})} \\
    \text{Weak inversion region: }&  \frac{\kappa I_{ds}}{\phi_t}
\end{align}
$\phi_t$ is the thermal voltage.
$\kappa$ is the gate coupling coefficient that is 1 in strong inversion and approximately between 0.4 to 0.7 in weak inversion.
This $\kappa$ for nanowire in weak inversion is believed to be close to 1 since the aqueous capacitance is far larger than the capacitance of silicon substrate.
Based on the equation, we believe that the $gm$ of nanowire can be determined by its $I_{ds}$.

\subsection{Proposed Measurement Method and Read-out circuit}
According to the last section, the $gm$ of mismatched nanowire devices can be adjusted to the same value by controlling their $I_{ds}$.
In this works, the changes of the biomolecule concentration is taken as an equivalent $\Delta V_G$.
Thus, a $\Delta V_G$ should produce same current response ($\Delta I_{ds}$) for mismatched device whose $gm$ are same.

This method is realized by our read-out circuit.
The circuit first performs DC sweep (Fig.\ref{fig:idvg}(a)) to find the $I_{ds}$-$V_{gs}$ curve of each device under test (DUT) (FIg.\ref{fig:idvg}(b)).
The $I_{ds}$-$V_{gs}$ curve of DUT is then transformed into $gm$-$I_{ds}$ curve (Fig.\ref{fig:idvg}(c)).
With these curves, the corresponding $I_{ds}$ values are found.
During each measurement, DUTs are always initialized with their specific $I_{ds}$ by adjusting the $V_{gs}$ in the end of each measurement task.

\begin{figure}[!bht]
    \begin{minipage}[!htb]{0.4\linewidth}
        \centering
        \def\svgwidth{3cm}
        \fontsize{6}{15}\selectfont
        \input {Graphs/Exp.pdf_tex}
        \fontsize{10}{10}\selectfont
        \makebox[3cm][l]{\textbf{(a)}}
    \end{minipage}
    % \hfill
    \begin{minipage}[!htb]{0.6\linewidth}
        \begin{minipage}[!htb]{0.97\linewidth}
            \centering
            \def\svgwidth{4.3cm}
            \fontsize{6}{15}\selectfont
            \input {Graphs/Id_Vsf.pdf_tex}
        \end{minipage}
        \makebox[4.3cm][l]{\textbf{(b)}}
        \vfill
        \begin{minipage}[!htb]{1\linewidth}
            \centering
            \def\svgwidth{3.8cm}
            \fontsize{6}{15}\selectfont
            \input {Graphs/gbs_Id.pdf_tex}
        \end{minipage}
        \makebox[4.3cm][l]{\textbf{(c)}}
    \end{minipage}
    % \fontsize{10}{5}\selectfont
    \caption{\textbf{(a)} DC sweep. \textbf{(b)} $I_{ds}$-$V_{gs}$ curves. \textbf{(c)} $gm$-$I_{ds}$ curves.}
    \label{fig:idvg}
\end{figure}




\subsection{Architecture}
The constant current structures such as source follower have been applied to several works of ion-sensitive field-effect transistor(ISFET) \cite{J6}, which is a relative of SiNW.
A similar structure is presented here.
The structure can switch between two modes: DC-sweep mode (DC mode)(Fig.\ref{fig:mode}(a)) and Transient Measurement mode (Tr mode) (Fig.\ref{fig:mode}(b)).

Operation in DC mode is similar to Source follower.
Except the negative feedback doesn’t happen at source but gate through feedback loop circuit.
This mode devotes to set up nanowire in the beginning when the reference ion solution is given.

Tr mode is used after suitable gate voltage is found in DC mode.
In this mode, the feedback loop is removed and the tested solution is then given.
The difference of the concentration of biomolecule results in an equivalent $\Delta V_G$ at gate.
A variance of current is induced and is converted to output by a transimpedance(TIA) and a voltage amplifier.

\begin{figure}[!tb]

    \begin{minipage}[htb][3cm][t]{0.18\linewidth}
        \textbf{
        \centering
        \def\svgwidth{2.62cm}
        \fontsize{6}{15}\selectfont
        \input {Graphs/gvt.pdf_tex}
        }
    \end{minipage}
    \hfill
    \begin{minipage}[htb][3cm][t]{0.65\linewidth}
        \textbf{
        \centering
        \def\svgwidth{4.3cm}
        \fontsize{6}{15}\selectfont
        \input {Graphs/cvm.pdf_tex}
        }
    \end{minipage}
    \vfill
    \makebox[1\linewidth][l]{}
    \makebox[2.9cm][l]{\textbf{(a)}}
    \makebox[4.3cm][l]{\textbf{(b)}}
    \caption{Circuit block diagram of \textbf{(a)} DC mode \textbf{(b)} Tr mode.}
    \label{fig:mode}
\end{figure}

\section{Circuit Implementation}
Fig.\ref{fig:NMS} shows the circuit schematic.
DC mode and Tr mode shared a common transimpedance(TIA), which is resistor-based because linearity is necessary for a wide input current range (from 10nA to 1uA).
This TIA block transduce $\Delta I_{ds}$ into a voltage signal while keeping the $V_{D}$ of the nanowire constant.
A controlling switch switches manually between integrated circuit and an external voltage source(Vb) that can memorize the voltage obtained by DC mode.

For DC mode, an open loop OP connected to the output of TIA.
It is used as a low pass filter with bandwidth $<$ 20Hz.
It introduces only DC or low frequency signal into the feedback loop.
Besides, the low frequency dominant pole it creates can keeps the feedback loop stable when sometimes the large transconductance of nanowire increases total loop gain a lot.

For the Tr mode, the output of TIA connected to an analog subtractor which is for shifting the offset voltage from $V_{Ref}$ to $V_z$.
It is then followed by a non-inverting resistor-based amplifier.
This amplifier has amplification rate of 100.
It amplifies the voltage difference between the output volatge of TIA and the reference voltage.

\begin{figure}[!htb]
    \textbf{
        \centering
        \def\svgwidth{8.0cm}
        \fontsize{6}{15}\selectfont
        \input {Graphs/NMS.pdf_tex}
    }
    \caption{Schematic of the read-out circuit}
    \label{fig:NMS}
\end{figure}


\section{Chip Measurement and Results of the Method}

\begin{figure}[!htb]
        \centering
        \includegraphics[width=0.8\linewidth]{Graphs/chip.jpg}
    \caption{Chip fabricated by TSMC 0.35$\mu$ process.}
    \label{fig:chipphoto}
\end{figure}

Fig.\ref{fig:chipphoto} is the photo of our read-out circuit.
The circuit is fabricated with the TSMC 0.35um process and has 8 units of read-out circuit on a single chip.
Its characteristics are shown in Table.\ref{tb:spec}.
\begin{table}[htb]
    {\fontfamily{}\fontsize{6}{10}\selectfont
    \centering
    \label{my-label}
    \begin{tabular}{l|cp{1.8cm}}
        \hline
        \hline
        VDD & 3.3v \\
        \hline
        Power Consumption & 1.48mW\\
        \hline
        Size & 1.57 x 1.57 mm$^2$ \\
        \hline
        \multicolumn{2}{c}{DC sweep mode}\\
        \hline
        \hline
        SiNW transconductance Range ($gm$) & $3\mu \sim 20\mu$\\
        \hline
        SiNW Bias Current Range ($I_{ds}$) & $1\mu A \sim 50\mu A$\\
        \hline
        SiNW Gate Voltage Range ($V_{gs}$) & $0.45v \sim 3v$ \\
        \hline
        \multicolumn{2}{c}{Transient measurement mode}\\
        \hline
        \hline
        Input current to Output voltage gain & $8.9M$ \\
        \hline
        $\Delta I_{ds}$ detecing Range &
            \begin{tabular}{@{}c@{}}
                $2.8n A \sim 5.3\mu A$ \\ $-15\mu A\sim -2.8n A$
            \end{tabular}
        \\
        \hline
         Maximal Signal Detection Speed(Hz) & $7.5k$\\
        \hline
         Maximal Input Reffered Noise(mV) & $0.3nA$ \\
    \end{tabular}
    \caption{Specification Summary}
    }
    \label{tb:spec}
\end{table}

The circuit of DC mode is able to measure device with $gm$ from $3\mu$ to $20\mu$.
The lower bound of this range can be further decreased.
An input mismatch problem occurs in the open loop OP due to fabrication variance.
It can be solved in the future by improving the layout or modifying the structure.
The Tr mode has total amplification rate of $8.9M$ ($G_{TIA} \times A_{amp}$).
It also has a very low input-referred noise.
For our nanowire, the maximal noise is $0.3nA$ when referred to $I_{ds}$ (Fig.\ref{fig:noise}), which is far lower than the equivalent $\Delta I_{ds}$ that biomolecules can induce.
Besides, the fastest signal speed it can detect is $7.5kHz$, which is enough for most biological experiments.

\begin{figure}[!htb]
        \centering
        \includegraphics[width=1\linewidth]{Graphs/Noise.png}
    \caption{The power spectral density of input referred noise. The 60Hz noise comes from the environment and working machines. It can be further reduce by adopting better method of experiment or equipments.}
    \label{fig:noise}
\end{figure}

In Fig.\ref{fig:dV}, the results of the proposed measurement method is presented.
The biomolecule testing solution is a pH buffer and the reference solution is double-distilled water.
The method is applied to two nanowire devices with device mismatch.
There $I_{ds}$-$V_{gs}$ curves are obtained (Fig.\ref{fig:dV}(a)) and $gm$-$I_{ds}$ curves are computed (Fig.\ref{fig:dV}(b)).
The Fig.\ref{fig:dV}(c) shows that the problem is mitigate.
Two devices exhibit similar responses with error less than 8\%.
This error should be experimental error.






\begin{figure}[!htb]

    \makebox[0.5\linewidth][l]{\textbf{(a)}}
    \makebox[0.5\linewidth][l]{\textbf{(b)}}
    \begin{minipage}[!htb]{0.5\linewidth}
        \centering
        \def\svgwidth{4cm}
        \fontsize{6}{8}\selectfont
        \input {Graphs/f6-25a.pdf_tex}
        % \fontsize{8}{6}\selectfont
    \end{minipage}
    \hfill
    \begin{minipage}[!htb]{0.5\linewidth}
        \centering
        \def\svgwidth{5.1cm}
        \fontsize{6}{8}\selectfont
        \input {Graphs/f6-25b.pdf_tex}
        % \fontsize{8}{6}\selectfont
    \end{minipage}
    \vfill
    \makebox[1\linewidth][l]{\textbf{(c)}}
    \begin{minipage}[!hb]{1\linewidth}
        \centering
        \def\svgwidth{9cm}
        \fontsize{8}{10}\selectfont
        \input {Graphs/f6-25c.pdf_tex}
        % \fontsize{8}{6}\selectfont
    \end{minipage}
    \caption{Proposed measurement method applied to two mismatched devices (nw1-2 and nw2-1).
                \textbf{(a)} Their $I_{ds}$-$V_{gs}$ curves.
                \textbf{(b)} Their $gm$-$I_{ds}$ curves.
                \textbf{(c)} The output responses when the testing solution (pH6 solution) is added into reference solution (double-distilled water).
                }
    \label{fig:dV}
\end{figure}

\section{Conclusion}
In summary, our method and circuit mitigate the device mismatch problem.
The circuit can be improved in the future by introducing digital circuit.
In that way, the $gm$-$I_{ds}$ curve obtaining and mode deciding can be automatic, and a systematical measurement architecture can be built.

\bibliographystyle{IEEEbib}
\bibliography{references}

\end{document}
